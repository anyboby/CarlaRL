%%%%%%%%%%%%%%%%%%%%%%%%%%%%%%%%%%%%%%%%%%%%%%%%%%%%%%%%%%%%%%%%%%%%%%%%%%%%%%%%
%2345678901234567890123456789012345678901234567890123456789012345678901234567890
%        1         2         3         4         5         6         7         8

\documentclass[letterpaper, 10 pt, conference]{ieeeconf}  % Comment this line out if you need a4paper

%\documentclass[a4paper, 10pt, conference]{ieeeconf}      % Use this line for a4 paper

\IEEEoverridecommandlockouts                              % This command is only needed if 
                                                          % you want to use the \thanks command

\overrideIEEEmargins                                      % Needed to meet printer requirements.

% See the \addtolength command later in the file to balance the column lengths
% on the last page of the document

% The following packages can be found on http:\\www.ctan.org
%\usepackage{graphics} % for pdf, bitmapped graphics files
%\usepackage{epsfig} % for postscript graphics files
%\usepackage{mathptmx} % assumes new font selection scheme installed
%\usepackage{times} % assumes new font selection scheme installed

%%%%%%%%%% my includes %%%

\usepackage{url}      % needed for ieeetran bib style
\usepackage{graphicx}

% possible subfigure packages
%\usepackage{subfigure}
%\usepackage[caption=false,font=footnotesize]{subfig}

\usepackage[colorinlistoftodos, german]{todonotes} % Option 'disable' entfernt alle ToDos

\usepackage[utf8]{inputenc}

\usepackage[font=footnotesize]{caption}
\usepackage[font=footnotesize]{subcaption}
\newtheorem{thm}{Theorem}[section]
\newtheorem{defn}[thm]{Definition}


\usepackage{hyperref}

%\usepackage[style=plain,citestyle=numeric,bibstyle=numeric,sorting=none,url=false,doi=false,isbn=false]{biblatex}

%%%%%%%%%%%%%%%%%%%%%%%%%%


\title{\LARGE \bf
Preparation of Papers for IEEE Sponsored Conferences \& Symposia*
}

\author{Florian Kuhnt$^{1}$ and J. Marius Z\"ollner$^{1}$ % <-this % stops a space
\thanks{$^{1}$The authors are with FZI Research Center for Information Technology, Haid-und-Neu-Str. 10-14, 76131 Karlsruhe, Germany
%        {\tt\small \{kuhnt, zoellner\}@fzi.de}}%
        {\tt\small \{kuhnt, zoellner\}@fzi.de}}%
}        
        
\begin{document}



\maketitle
\thispagestyle{empty}
\pagestyle{empty}


%%%%%%%%%%%%%%%%%%%%%%%%%%%%%%%%%%%%%%%%%%%%%%%%%%%%%%%%%%%%%%%%%%%%%%%%%%%%%%%%
\begin{abstract}

This electronic document is a ÒliveÓ template. The various components of your paper [title, text, heads, etc.] are already defined on the style sheet, as illustrated by the portions given in this document.

\end{abstract}


%%%%%%%%%%%%%%%%%%%%%%%%%%%%%%%%%%%%%%%%%%%%%%%%%%%%%%%%%%%%%%%%%%%%%%%%%%%%%%%%
\section{Introduction}

This template provides authors with most of the formatting specifications needed for preparing electronic versions of their papers. All standard paper components have been specified for three reasons: (1) ease of use when formatting individual papers, (2) automatic compliance to electronic requirements that facilitate the concurrent or later production of electronic products, and (3) conformity of style throughout a conference proceedings. Margins, column widths, line spacing, and type styles are built-in; examples of the type styles are provided throughout this document and are identified in italic type, within parentheses, following the example. Some components, such as multi-leveled equations, graphics, and tables are not prescribed, although the various table text styles are provided. The formatter will need to create these components, incorporating the applicable criteria that follow.

Example Citation: \cite{Barth2008}.

\subsection{Problem specification}
\subsection{Why RL? What is our goal/motivation?}

\section{Related Work}

\subsection{“Human-level control through deep reinforcement learning” (2015)}
\subsection{“CARLA: An Open Urban Driving Simulator”}
\subsection{Our contribution to the field}

\section{Concept}

\subsection{CarRacing - Understand and select algorithms}
\subsubsection{Preprocessing}
\subsubsection{DQN}
\subsubsection{A3C}
\paragraph{Model architecture}
\paragraph{Training details}
\subsubsection{DDPG}
\paragraph{Model architecture}
\paragraph{Training details}

\subsection{CARLA}
\subsubsection{State representation}
\subsubsection{Used sensors}
\subsubsection{Implementing the reward function}


\section{Evaluation}

\subsection{Results}
\subsection{Comparison algorithms/reward functions}

\section{Conclusions}


\addtolength{\textheight}{-12cm}   % This command serves to balance the column lengths
                                  % on the last page of the document manually. It shortens
                                  % the textheight of the last page by a suitable amount.
                                  % This command does not take effect until the next page
                                  % so it should come on the page before the last. Make
                                  % sure that you do not shorten the textheight too much.

%%%%%%%%%%%%%%%%%%%%%%%%%%%%%%%%%%%%%%%%%%%%%%%%%%%%%%%%%%%%%%%%%%%%%%%%%%%%%%%%



%%%%%%%%%%%%%%%%%%%%%%%%%%%%%%%%%%%%%%%%%%%%%%%%%%%%%%%%%%%%%%%%%%%%%%%%%%%%%%%%



%%%%%%%%%%%%%%%%%%%%%%%%%%%%%%%%%%%%%%%%%%%%%%%%%%%%%%%%%%%%%%%%%%%%%%%%%%%%%%%%

%References are important to the reader; therefore, each citation must be complete and correct. If at all possible, references should be commonly available publications.










\newpage

\bibliographystyle{IEEEtran}
\bibliography{IEEEabrv,04_mendeley-export/library}



\end{document}
